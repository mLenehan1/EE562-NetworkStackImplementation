The Red-Black Tree is an implementation of the Binary Search Tree which
maintains balance. Nodes of a Red-Black tree are either coloured red or black,
hence the name. By ensuring that there are never two directly connected red
nodes, rotate operations ensure that the tree is always balanced.

The concept for this type of binary search tree was instroduced by Rudolf Bayer
in 1972\cite{wpedia}. They are particularly useful in real-time applications, and as such
were included in the implementation of the ``Completely Fair Scheduler'' in the
Linux Kernel in version 2.6.23.

The main advantage of using Red-Black trees is the time complexity. Red-Black
trees have a time complexity of O(logN) for insertion and deletion operations,
improving upon the typical complexity of O(N) for more basic Binary Search Tree
implementations\cite{wpedia}.

For each node in a tree there is a set colour, up to one parent node, and up to
two child nodes. The root node if a red-black tree is always coloured black,
along with any null nodes (i.e. empty/leaf nodes). Two red nodes can not be
directly connected\cite{geeksIntro}. Each left child node in a tree must
have a value less than the parent node, and each right child node in a tree must
have a value greater than the parent node. Maintaining balance within the tree
means that the case of the root node being the lowest value in a constantly
incrementing tree, i.e. all child nodes are in the right branch, cannot occur.
This balancing means that the time complexity of traversing the tree does not
increase past O(logN).
