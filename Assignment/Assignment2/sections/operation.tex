The main feature of a red-black tree is it's ability to maintain balance. It
does this through a number of operations. On adding a new value to to tree, an
Insertion operation is used to define where within the tree the new value will
be placed. On deleting a value from the tree, the tree must be restructured in
order to allow any child node of the deleted node to remain connected to the
rest of the tree. Recoloring is a technique which is unique to Red-Black Trees,
and allows for a tree to be rebalanced without using a rotation operation. If
recoloring is not successful, rotations are used following these operations to
ensure that balance is maintained within the tree.

\subsection{Insertion}

Insertions are done based on the value of the incoming node. The incoming node
value is compared agains the root node value. If the incoming node value is
greater than the root node value, the comparison is repeated with the root nodes
right child node. If the incoming node value is less than the root nodes value,
the comparison is repeated with the root nodes left child node\cite{geeksInsert}
. This operation
is typically called recursively in order to traverse the tree until the point
that there is no left or right child node with which to compare, at which point
the incoming node can be placed in the tree.

Following the insertion of the incoming node to the tree, the tree must be
rebalanced. This is to maintain the traversal speed of the tree. Rebalancing is
done via either recoloring or rotation, as described below.

\subsection{Deletion}

Deletion within a Red-Black tree involves removing the specified value from the
tree. The specified value is compared to the root node value, and, if greater
than the root node value, a comparison is made with the right child node value,
and if less than the root node value, a comparison is made with the left child
node value. This operation is called recursively until the specified node value
is found. This node is then deleted from the tree.

If the specified node value is a leaf node, or a node with one child, then this
deletion is relatively simple. If the node has one child, then the child node
replaces the deleted node within the tree and a recolouring or rotation must be
performed. If the node has no children, i.e. is a leaf node, then the node is
simply deleted\cite{geeksDelete}.

For any node deletion in which the specified node is not a leaf node, or a node
with only one child, i.e. the specified node is ``internal'', the order of all
child nodes must be retained, and after deletion must be reordered into the
existing tree\cite{geeksDelete}. This is, therefore, a much more complex operation.

\subsection{Tree Balancing}

Tree balancing is required in order to maintain the traversal time complexity
within a red-black tree. The two techniques used in tree balancing are
``Recoloring'' and ``Rotation''. Recoloring involves changing the colours
associated with the nodes in the tree, whereas rotation involves moving the
nodes of the tree, replacing the existing root node.

\subsubsection{Recoloring}

There are a number of cases for recoloring of nodes within a red-black tree. As
the root node is required to be black, and two red nodes cannot be adjacent,
the recoloring must be done in such a way as to maintain these rules.

Any new node being inserted to the tree is initially coloured red. The parent
node colour is then checked. If this node is red, the rule stating that two red
nodes cannot be directly connected is being violated. The parent node color is
changed to black, and the colour of the parents parent (i.e. the grandparent),
is changed to red. Following this, rotations may be required to make a valid
red-black tree, for example if the grandparent node is the root of the tree
(which must be black)\cite{geeksInsert}.

\subsubsection{Rotation}

Rotations are used within the red-black tree to rebalance the tree without
modifying the overall order of the elements in the tree. When a node is
inserted, recoloring is first performed, however, if this does not result in a
valid, balaced, red-black tree, a rotation operation must be completed.

Following recolouring, if the root node is red, a left or right rotation must
occur. If the newly added node is in the left branch, a right rotation occurs,
and if in the right branch, a left rotation occurs\cite{geeksInsert}.
These rotations may be
requied to be done recursively in order to acquire a valid and balanced red-black tree.
