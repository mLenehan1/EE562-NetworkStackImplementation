Red-Black Trees offer traversal time complexity in the order O(logN). While the
implementation may be complex, the benefits of the red-black tree are clear, and
its benefits in real-time applications are why it was included in the Linux
Kernel's ``Completely Fair Scheduler''.

The red-black tree code shows how the tree can be used to sort values, while
maintaining balance, and the time complexity of the tree. An issue was
experienced in the code, when using the randomly generated numbers the tree
would overwrite each number in the tree with the last input number, resulting in
only one number (the final input number) in the completed tree. As no changes
were made to the structure of the tree's code, this issue was unable to be
resolved.
