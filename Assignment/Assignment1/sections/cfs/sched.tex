\subsection{Scheduling Algorithms}

Scheduling algorithms are a way for the kernel to provide CPU resources to
processes in a balanced way. The original scheduling algorithm used by the
Linux kernel based it's scheduling on the time a process was running for.
Processes which require user input require more time on the CPU, as there cannot
be a perceptible delay in execution of user commands. This original scheduling
algorithm was of constant time complexity, and as such was named the Linux O(1)
scheduler\cite{pabla_2009}.

The Completely Fair Scheduler was introduced as an improvement upon the O(1)
scheduler, improving the time complexity to O(logn), and reducing the scheduling
codebase by approximately 700 lines\cite{molnar_2007}.
