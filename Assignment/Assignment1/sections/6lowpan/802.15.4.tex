\subsection{802.15.4}

The IEEE 802.15.4 standard gives specification for Low-Rate Wireless Personal Area
Networks (LR-WPAN). It defines specifications for Layer 1 and 2 protocols (Physical
and Data Link layers respectively)\cite{ott_2012}. This standard aims to give a platform
on
which higher level protocols can be based. The specification allows for data
rates as low as 20kbps, and is intended for use on low-power devices, allowing
for extended battery life, due to the low power consumption for data
transfers\cite{electronics_notes}.

The initial release of this standard occurred in 2003, giving specifications for
two physical layer implementations, one lower frequency band, and one upper,
with the upper frequency band being 2.4GHz. Updates have included improvements
in the available data rates, modulation schemes, the addition of ultra-wide band
frequency ranges, MAC support, and, with the latest release (802.15.4g),
physical layer implementations for ``smart neighbourhood
networks''\cite{electronics_notes}.

There are a number of higher level protocols which make use of the 802.15.4
standard, including MiWi, 6LoWPAN, WirelessHART, and ISA100.11a. Each of these
protocols uses the standard for different purposes, such as mesh networking or
manufacturing automation. One of the most well known of these protocols, Zigbee,
uses the original 802.15.4 specification, however, due to it's non-open source
implementation, it is assumed that there will not be an implementation added to
the Linux kernel\cite{ott_2012}.

IEEE 802.15.4 offers data rates up to 250kbps, with power consumption as low as
17mA during transmission, and 19mA during receiving. These specifications
offer advantages for wireless sensors, and other devices, making it a good
choice for the establishment of Personal Area Networks\cite{ott_2012}.
