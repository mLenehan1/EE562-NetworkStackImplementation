\subsection{Userspce}

As previously mentioned, in userspace, the user application data is found. This
is also where management services are found. These services include
the WPA Supplicant, nl80211 configuration utilities, and access point
authenticators\cite{chou_2015}.

WPA Supplicant supports both WPA and WPA2, and is run as a background process,
controlling wireless connections. wpa\_supplicant, the implemented Linux kernel
process, can be configured using a text configuration file on the users machine.
This process controls aspects of the wlan driver, such as the roaming and 802.11
authentication and association\cite{wpa_supplicant}.

``iw'' is a configuration utility for nl80211, which is a netlink interface,
providing management between the kernel and user spaces. iw can be used to
connect to an access point, modify transmission configuration such as
transmission bitrates or power, and get the status of the current network
link\cite{iw}.

``hostapd'' is an authenticator for 802.11AP and 802.1X/WPA/WPA2/EAP/RADIUS. It
is used to configure the wireless infrastructure, including authenticating
clients, and setting encryption keys. hostapd provides authentication and
encryption settings for MAC filtering, ignoring broadcast packets, enabling
WPA/WPA2, and key management algorithms for WPA/WPA2\cite{hostapd}.
