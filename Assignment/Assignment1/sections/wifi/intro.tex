Wireless networking in Linux is essentially broken into three sections, these
sections, as discussed in detail below, are the userspace, kernel space, and
device driver level. Each processes data at different stages of transmission.
The userspace deals with the user applications, and also with some management
processes. The kernelspace deals with network sockets, and network protocols. On
the management side it deals with MAC implementation. Finally, at the device
driver level, the information of higher levels is combined and placed in a queue
for transmission\cite{chou_2015}.
