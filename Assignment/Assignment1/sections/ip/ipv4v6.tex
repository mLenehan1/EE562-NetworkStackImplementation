There are a number of differences which have been made with the implementation
of IPv6. One of the most notable is the change from 32 bit IP addresses to 128
bit IP addresses. This change supports and increase in the total number of
addresses from $2^{32}$ addresses to  $2^{128}$ addresses\cite{v4v6}. This increase allows
for the limiting of routing table sizes, and helps to deal with the exhaustion
of available IPv4 addresses\cite{ietftools}. The Internet of Things will benefit
greatly from
this increase in available addresses, as all connected devices can be given
unique addresses.

Categorization of addresses has changed from the IPv4 model of unicast,
multicast, and broadcast, to the IPv6 model of unicast, multicast, and anycast.
With broadcast, all nodes on a network receive the sent packets, even those who
do not require the message. This creates load on the network. Anycast sends a
packet to the closest/least expensive node of a group of nodes with the same
destination address in the routing table\cite{ipcisco}.
