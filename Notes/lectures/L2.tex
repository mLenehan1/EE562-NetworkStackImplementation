\section{Kernel Issues}

\subsection{Kernel Programming Issues}

\begin{itemize}
	\item Library functions not available
	\item Limited Stack - Small, fixed-size stack for each proc
	\begin{itemize}
		\item kmalloc args 1=Size of block, 2=how kmalloc works
		\item GFP\_Kernel - kmalloc puts curnt. proc to sleep, waits for
			page when called in low mem situations
		\item Funciton using GFP must be reentrant, otherwise use
			GFP\_Atomic
	\end{itemize}
	\item Kernel functions must be rentrant
	\item Linux kernel fully preemptible
	\begin{itemize}
		\item kernel proc can be preempted by > priority proc
		\item Should avoid global static vars in network code.
		\item Need to use spinlock/semaphores to protect access to
			global structs
	\end{itemize}
	\item Linux supports Symmetric Multiprocessor (SMP) architectures
	\begin{itemize}
		\item Mutli. processors share common mem, potential issues
		\item Harder debugging
		\item Bugs cause catastrophic failure, require reboot
	\end{itemize}
\end{itemize}

\subsection{Memory Allocation in the Kernel}

\begin{itemize}
	\item Slabs \ldots
\end{itemize}

\subsection{Kernel Scheduling}

\begin{itemize}
	\item Multi. Scheduling Classes
	\item Various policies within class
	\item Higher priority class served first
	\item Tasks can migrate between classes, policies, CPUs
	\item Completely Fair Scheduler
	\begin{itemize}
		\item Decides which process is executed next
		\item Uses red-black tree to decide on proc.
		\item Implements Weighted Fair Queuing
	\end{itemize}
\end{itemize}

\begin{table}[h]
	\centering
	\begin{tabular}{c|c}
		Class & Policies \\
		\hline
		Stop & none \\
		Deadline & SCHED\_DEADLINE \\
		Real Time (RT) & SCHED\_FIFO, SCHED\_RR \\
		Fair (CFS) & SCHED\_NORMAL, SCHED\_BATCH, SCHED\_IDLE \\
		Idle & none\\
	\end{tabular}
\end{table}

\begin{table}[h]
	\centering
	\begin{tabular}{c|c}
		Class & Features \\
		\hline
		Stop & SMP-only, cannot be preempted, preemts all \\
		Deadline & For periodic RT tasks \\
		Real Time (RT) & for Posix ``real-time`` (low latency) tasks \\
		Fair (CFS) & All other system tasks \\
		Idle & Nothing to do - enter low power mode \\
	\end{tabular}
\end{table}

\subsection{Symmetric Multiprocessing (SMP)}

\begin{itemize}
	\item Like parallel processing
	\item Multi resources share single bus with multi. processors
	\item Contention for resources
	\begin{itemize}
		\item N cores not N times faster than 1
		\item Need to lock shared resources when in use
	\end{itemize}
\end{itemize}

\subsection{Reentrancy and Preemption}

Preemption
\begin{itemize}
	\item Stopping code to run other code
\end{itemize}

Reentrancy
\begin{itemize}
	\item Preemting process might call preemted code
	\begin{itemize}
		\item Preemnted code must support multi. execution instances
	\end{itemize}
\end{itemize}

\subsection{Writing reentrant code}

\begin{itemize}
	\item Avoid global vars, use automatic vars/kmalloc instead
	\item Use spinlock/semaphors to ensure global var access is atomic
	\item Do not call other functions, unless they are reentrant
	\item Accessing HW causes issues, use spinlock/semaphores to lock out
		other proc.
	\begin{itemize}
		\item Causes bad RT performance
		\item Fix - Disable interrupts
		\item Most interrupt handlers do this
	\end{itemize}
\end{itemize}

\subsection{Spinlock}

\begin{itemize}
	\item Crude method to protect data struct
	\item Single int field as lock
	\begin{itemize}
		\item Proc withing to access protected region must check lock
			val
		\item If 1: proc. retries (spins) in tight loop of code
		\item If 0: proc. enters region, sets value to 1
	\end{itemize}
	\item Accessing mem location must be atomic - cannot be interrupted
	\item Winning proc is random/arbitrary
\end{itemize}

\subsection{Semaphores}

\begin{itemize}
	\item Protect critical regions of code/data structs
	\item Queueing mechanism to give order to access
	\item Count
	\begin{itemize}
		\item Num of proc waiting for resource
		\item +: resource available
		\item -/0: procs waiting
		\item Initially set to 1
		\item Requesting resource dec. count
		\item Proc finished - inc. count
		\item Abs val of count = num of sleepers
	\end{itemize}
	\item Wait
	\begin{itemize}
		\item Queue in which sleeping procs stored
	\end{itemize}
\end{itemize}

\subsection{HW Interrupts}

\begin{itemize}
	\item I/O Interrupts
	\begin{itemize}
		\item Keyboard Press, USB bus has data, pkt arrives in NIC
	\end{itemize}
	\item Timer Interrupts
	\begin{itemize}
		\item Maintaining System Clock, Executing sched alg
	\end{itemize}
	\item Interprocesspr Interrupts (IPI)
	\begin{itemize}
		\item Only in SMP systems
		\item On ARM processors
		\begin{itemize}
			\item IPI\_RESCHEDULE ``Rescheduling interrupts``
			\item IPI\_CALL\_FUNC ``Function call interrupts``
			\item IPI\_CPU\_STOP ``CPU stop interrupts``
			\item IPI\_CPI\_CRASH\_STOP ``CPU stop (for crash dump)
				interrupts``
			\item IPI\_TIMER ``Timer broadcast interrupts''
			\item IPI\_IRQ\_WORK ``IRQ work interrupts''
			\item IPI\_WAKEUP ``CPU wake-up interrupts''
			\item Used for e.g. power management, moving procs
				between CPUs
		\end{itemize}
	\end{itemize}
\end{itemize}

\subsection{Timer Interrupts}

\begin{itemize}
	\item Triggered by Programmable Interval Timer/High Precision Event
		Clock
	\begin{itemize}
		\item ISR (interrupt service routine) implement jiffy clock,
			invoke proc scheduling
	\end{itemize}
	\item SMP systems have local CPU timers
	\begin{itemize}
		\item Used for profiling kernel code, timing procs.
	\end{itemize}
\end{itemize}
