\section{Binary Trees}

\subsection{Binary Trees}

\begin{itemize}
	\item Sorted lists needed throughout kernel
	\begin{itemize}
		\item Queueing disciplines
		\item Process scheduling
		\item Inode management
		\item Routing tables
		\item etc,
	\end{itemize}
	\item Various types
	\begin{itemize}
		\item Radix Tree (not necessarily binary)
		\begin{itemize}
			\item Used e.g. in the device driver for the Mellanox
				Connect X-3 10 Gbe NIC
		\end{itemize}
		\item Red-Black Tree
		\begin{itemize}
			\item Used in IPv4 FIB (forwarding information base)
		\end{itemize}
		\item Used in IPv6 FIB
	\end{itemize}
\end{itemize}

\subsection{Balanced Tree}

\subsection{Unbalanced Tree}

\subsection{Tree Rotation}

\begin{itemize}
	\item A tree can be rebalanced using (multi). rotations
\end{itemize}

\subsection{Rebalanced Tree}

\subsection{Aside: Trees and Tries}

\begin{itemize}
	\item The word ``trie'' is the second syllable of ``retrieval''
	\item Also called prefix tree
	\item Each vertex represents
	\begin{itemize}
		\item Location for value or key (in standard tree)
		\begin{itemize}
			\item e.g. the inode of a file
		\end{itemize}
		\item A matched pattern (in a trie)
		\begin{itemize}
			\item e.g. ``Cam*'' matches
				``Cam'',``Came'',``Camel''\ldots
		\end{itemize}
	\end{itemize}
\end{itemize}

\subsection{Trie Compression}

\begin{itemize}
	\item Path Compressions
	\begin{itemize}
		\item If a node has only one child, merge node and child
		\begin{itemize}
			\item Reduces tree depth
			\item Increases node storage required
		\end{itemize}
		\item Level compression
		\begin{itemize}
			\item A node can have  $2^r$ children instead of a max
				of 2
			\item r is the radix of the trie
		\end{itemize}
	\end{itemize}
\end{itemize}

\subsection{Patricia Tree}

\begin{itemize}
	\item A redix tree is a trie featuring path compression and level
		compression.
	\item Radix tree with 1/both of these properties
	\begin{itemize}
		\item Radix is 1
		\item PATRICIA algorithm
		\begin{itemize}
			\item ``Practical Algorithm To Retrieve Information
				Coded in Alphanumeric''
		\end{itemize}
	\end{itemize}
\end{itemize}

\subsection{Red-Black Tree}

\begin{itemize}
	\item Each node is either red/black
	\begin{itemize}
		\item Root node always black
	\end{itemize}
	\item All leaves (empty terminal nodes) are black - black rectangles
	\item Num of black nodes traversed in travelling from any node to any of
		its descendant leaves is the same
	\begin{itemize}
		\item Prevents tree imbalance, requires maintenance
		\item Following insertions/deletions, repair tree
		\begin{itemize}
			\item Repaint nodes to retain colour properties
			\item If repainting does not work use rotations to
				``fix'' the tree
		\end{itemize}
	\end{itemize}
\end{itemize}
