\section{Socket Buffers}

\subsection{Socket Buffers in Linux}

\begin{itemize}
	\item sk\_buff struct defines doubly-linked list of socket buffers with
		extra struct to allow head of list to be found quickly
	\item Various functions available to manage socket buffers
	\item Data space created in Linux sk\_buff comprises:
	\begin{itemize}
		\item Headroom
		\item Data
		\item Tailroom
	\end{itemize}
	\item Idea - Pkt being tx, amnt space reserced eq. that req. at L2, even
		though sk\_buff being created in Transport Layer
	\item Extra head/tailroom reserved for anticipated req. of lower-layer
		headers and trailers.
	\item When pkt rx from higher layer, head/tailroom encroached upon to
		create extra mem. space required.
	\item When created by alloc\_skb(), sk\_buff is "all tail".
	\item skb\_reserve() is called in to reserve headroom
	\item Space for data created using skb\_push() (bite into headroom) or
		skb\_put() (bite into tailroom)
	\item sk\_buff will be queued in doubly-linked list by protocol handler
	\item Following atomic func. avvailable to manipulate such lists (each
		list assoc. with socket):
	\begin{itemize}
		\item skb\_queue\_head() - Places buffer at start of list
		\item skb\_queue\_tail() - Places buffer at end of list
		\item skb\_dequeue() - Takes first buffer from list, returns
			pointer to sk\_buff or Null if queue empty
		\item skb\_insert(), skb\_append() - Place sk\_buff before/after
			specific buffer in list. Useful for pkt resequencing in
			TCP
		\item kfree\_skb() - Releases a buffer
		\item skb\_unlink() - Extracts sk\_buff from the list in which
			it is stored (w/o needing to know list identity) does
			not free it.
		\item skb\_clone(), skb\_copy() - Makes a copy of an sk\_buff
		\begin{itemize}
			\item Former doesn't copy data area (thus has much lower
				OH than skb\_copy())
			\item Disadvantage is that data area in cloned buffer is
				read-only
		\end{itemize}
	\end{itemize}
\end{itemize}

\subsection{Non-Linear Socket Buffers in Linux}

\begin{itemize}
	\item Two forms of non-linear buffers supported:
	\begin{itemize}
		\item Paged Buffers
		\begin{itemize}
			\item Large buffers may be neededif L2 supports e.g.
				64kB frames, or if using some HW acceleration
				methods
		\end{itemize}
		\item Fragmented Buffers
		\begin{itemize}
			\item When pkt is being reassembled from its fragments,
				data is assembled using linked list, to avoid
				buffer copying
		\end{itemize}
	\end{itemize}
	\item data\_len records pages in socket buffer, and so is 0 for
		"ordinary" linear buffers
\end{itemize}

\subsection{Hack for 64-bit Systems}

\begin{itemize}
	\item In moving from v2.6.21 -> .22 of kernel, hack used to reduce size
		of skb\_tail and skb\_end to 4 from 8 bytes.
	\begin{itemize}
		\item Change never documented.
	\end{itemize}
	\item Tail and End are now offsets, specifically the no. of bytes
		between head and tail of bff.
	\item transport\_header, network\_header, and mac\_header fields in
		struct now offsets, rather than pointers.
	\item No longer safe in versions > 2.6.22 to directly access these
		fields
	\begin{itemize}
		\item Inline functions defined to compute relevant values safely
	\end{itemize}
\end{itemize}
