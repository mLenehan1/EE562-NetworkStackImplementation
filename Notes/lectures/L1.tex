\section{Intro}

\subsection{Userspace vs Kernel}

\begin{itemize}
	\item Modern processors have 2 modes of operation
	\begin{itemize}
		\item Intel - Ring 0/3
		\item ARM - Supervisor/User
	\end{itemize}
	\item Code in Ring 0 > Priority than Ring 3
	\begin{itemize}
		\item More instructions from set
		\item More resource access
	\end{itemize}
	\item Allow for improved security
	\item System code = kernel
\end{itemize}

\subsection{Monolithic vs Microkernel}

\begin{itemize}
	\item Monolithic
	\begin{itemize}
		\item 1 multi-tasking prog. implements OS funcitons
		\begin{itemize}
			\item Add new functionality requires kernel rebuild
			\item Linux, FreeBSD
		\end{itemize}
	\end{itemize}
	\item Microkernel
	\begin{itemize}
		\item Most services run in userlan, min. code is privileged
		\item Enhances modularity/maintainability at expense of
			performance
		\item Mach, GNU Hurd, Minix 3
	\end{itemize}
	\item Hybrid Kernel/Macrokernel
	\begin{itemize}
		\item Most services run on microkernel on Ring 0
		\item Windows > NT
	\end{itemize}
\end{itemize}

\subsection{Loadable Modules}

\begin{itemize}
	\item Dynamically loaded at runtime
	\item Access kernel structures/methods which have been exported
\end{itemize}

\subsection{OSI Model Limitations}

\begin{itemize}
	\item OSI Model is approx. guide
	\begin{itemize}
		\item Session/Presentation not in most networked apps
		\item Tunneling can result in multi. stacked L3 protocols
		\item Ethernet Swithces nominally operate at L2, supposed to be
			``data-link`` protocol layer.
		\item Control plane does not feature in model
	\end{itemize}
\end{itemize}

\subsection{Bit Rates and Packet Rates}

Packet Rates depend on:

\begin{itemize}
	\item Physical Bit Rate
	\item OH
	\begin{itemize}
		\item Extra bits + by lower layers/protocol features
	\end{itemize}
	\item Pkt length
	\item Physical BR vary from V.Low to V.High
	\item Useable BR < Nominal BR
	\begin{itemize}
		\item To inc. data transfer rate of X from Transport to higher
			layers, bit rate at phy layer must be \gg X
	\end{itemize}
	\item Net SW must be written so net equip can operate at wire speed
	\begin{itemize}
		\item HW and SW in net device rd/wr 2 pkts from link w/o dead time
	\end{itemize}
\end{itemize}

\subsection{Headers/Trailers}

\begin{itemize}
	\item Each pkt comprises of Protocol Data Unit and new layers head/tail
	\item Exception: Multiplexing, Fragmentation/Re-Assembly
\end{itemize}

\subsection{Buffer Management}

\begin{itemize}
	\item Memory space (``socket buffer``) req. to store data varies between
		layers
	\item Solutions:
	\begin{itemize}
		\item Allocate new buffer
		\begin{itemize}
			\item Buffer contents PBV, cp to larger buffer (with
				offset for header)
			\begin{itemize}
				\item Lots of cp ops.
			\end{itemize}
		\end{itemize}
		\item Oversized Linear Buffers (Linux)
		\begin{itemize}
			\item How big skt buff is at lowest level implemented
			\item Allocate buffer of this size, offset pointer to it
				by aggregate size of lower level headers
			\item Good performance, breaches layering principle
		\end{itemize}
		\item Linked Minibuffers (BSD Unix)
		\begin{itemize}
			\item Header/PDU/Tailer stored as linked list
			\item Head/Tail + by inserting minibuffers at start/end
				of list
			\item Memory-efficient, lower performance
		\end{itemize}
	\end{itemize}
\end{itemize}
